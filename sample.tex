\documentclass[submit]{ipsj-luatex}

\usepackage{graphicx}
\usepackage{url}

% カウンタ設定(最終原稿時に使用)
%\setcounter{巻数}{65}
%\setcounter{号数}{1}
\setcounter{volume}{65}
\setcounter{number}{1}
\setcounter{page}{456}

% 受付・採録日(最終原稿時に使用)
% \受付{2025}{8}{19}
% \採録{2025}{10}{15}

\begin{document}

\title{jlreqベースLuaTeX用IPSJ論文クラスの開発と評価}
\etitle{Development and Evaluation of IPSJ Paper Class for LuaTeX Based on jlreq}

\affiliate{IPSJ}{情報処理学会\\
Information Processing Society of Japan, Tokyo 101-0062, Japan}

\affiliate{UNIV}{情報処理大学\\
Information Processing University, Kanagawa 220-0012, Japan}

\paffiliate{NEWORG}{新組織研究所\\
Presently with New Organization Research Institute}

\author{情報 太郎}{Taro Joho}{IPSJ}[taro.joho@ipsj.or.jp]
\author{処理 花子}{Hanako Shori}{IPSJ,UNIV}
\author{学会 次郎}{Jiro Gakkai}{UNIV,NEWORG}[jiro.gakkai@neworg.ac.jp]

\begin{abstract}%
本稿では,jlreqクラスをベースとしたLuaTeX用IPSJ論文クラスファイルipsj-luatex.clsの
開発について述べる.従来のipsj.clsはpLaTeX環境を前提としており,LuaTeX環境での使用には
制限があった.本クラスファイルは,LuaTeX環境でのUnicode対応とOpenTypeフォントの活用を
可能にしながら,IPSJ論文誌の厳密なフォーマット要件を満たす.評価実験では,
既存のipsj.clsとの高い互換性を保ちつつ,モダンなLaTeX機能を活用できることを確認した.
\end{abstract}

\begin{jkeyword}
情報処理学会,LuaTeX,jlreq,クラスファイル,Unicode,OpenType
\end{jkeyword}

\begin{eabstract}%
This paper describes the development of ipsj-luatex.cls, a LuaTeX-compatible 
IPSJ paper class file based on the jlreq class. The conventional ipsj.cls 
was designed for pLaTeX environment and had limitations when used with LuaTeX. 
This class file enables Unicode support and OpenType font utilization in 
LuaTeX environment while satisfying the strict formatting requirements of 
IPSJ Journal. Evaluation experiments confirmed high compatibility with the 
existing ipsj.cls while enabling the use of modern LaTeX features.
\end{eabstract}

\begin{ekeyword}
      IPSJ, LuaTeX, jlreq, class file, Unicode, OpenType
\end{ekeyword}

\maketitle

\section{はじめに}

情報処理学会論文誌ジャーナルでは,従来からpLaTeX環境での論文執筆を支援するため,
ipsj.clsクラスファイルを提供している\cite{ipsj-style}.しかし,近年のLaTeX環境の
進歩に伴い,Unicode対応やOpenTypeフォントの活用が可能なLuaTeX環境での論文執筆の
ニーズが高まっている\cite{luatex-guide}.

本稿では,W3C日本語組版要件\cite{jlreq-w3c}に準拠したjlreqクラス\cite{jlreq-class}を
ベースとして開発したLuaTeX用IPSJ論文クラスファイルipsj-luatex.clsについて述べる.

\section{関連研究と既存システムの問題点}

\subsection{既存のipsj.clsの特徴と制限}

従来のipsj.clsは以下の特徴を持つ:
\begin{enumerate}
\item pLaTeX環境での動作を前提
\item A4縦型2段組レイアウト
\item 論文誌トランザクション固有の機能
\item 豊富なオプション体系
\end{enumerate}

しかし,以下の制限があった:
\begin{itemize}
\item Shift\_JIS文字コードへの依存
\item 限定的なフォント選択肢
\item Unicode文字への対応不足
\end{itemize}

\subsection{jlreqクラスの優位性}

jlreqクラスは以下の優位性を持つ:
\begin{itemize}
\item W3C日本語組版要件への準拠
\item LuaTeX環境での最適化
\item 豊富なカスタマイズオプション
\item 活発な開発コミュニティ
\end{itemize}

\section{提案手法}

\subsection{設計方針}

本クラスファイルの設計では以下の方針を採用した:

\begin{enumerate}
\item \textbf{互換性の確保}: 既存のipsj.clsとのAPIレベル互換性
\item \textbf{拡張性の確保}: モダンなLaTeX3機能の活用
\item \textbf{保守性の確保}: jlreqクラスの機能を最大限活用
\end{enumerate}

\subsection{実装アーキテクチャ}

\figref{fig:architecture}に実装アーキテクチャを示す.

\begin{figure*}[t]
\centering
\begin{minipage}{0.8\textwidth}
\centering
\rule{12cm}{6cm}
\caption{ipsj-luatex.clsの実装アーキテクチャ}
\ecaption{Implementation architecture of ipsj-luatex.cls}
\label{fig:architecture}
\end{minipage}
\end{figure*}

クラスファイルは以下の層で構成される:

\begin{description}
\item[オプション処理層] IPSJ固有オプションの処理
\item[jlreq連携層] jlreqクラスとの連携機能
\item[フォント設定層] LuaTeX用フォント設定
\item[レイアウト調整層] IPSJ仕様への適合
\item[コマンド定義層] IPSJ固有コマンドの実装
\end{description}

\section{実装}

\subsection{基本クラス設定}

jlreqクラスの読み込み時に,以下のIPSJ固有設定を適用する:

\begin{verbatim}
\LoadClass[
  paper=a4,
  fontsize=10pt,
  jafontsize=10pt,
  line_length=43zw,
  number_of_lines=46,
  column_gap=2zw,
  twocolumn,
  open_bracket_pos=nibu_tentsuki
]{jlreq}
\end{verbatim}

\subsection{フォント設定}

LuaTeX環境でのフォント設定は\tabref{tab:fonts}のように行う.

\begin{table}[tb]
\caption{フォント設定一覧}
\ecaption{Font configuration list}
\label{tab:fonts}
\centering
\begin{tabular}{|l|l|l|}
\hline
種別 & submitオプション & 本格組版 \\
\hline\hline
欧文明朝 & Latin Modern Roman & Times New Roman \\
欧文Sans & Latin Modern Sans & Arial \\
欧文等幅 & Latin Modern Mono & Courier New \\
和文明朝 & Noto Serif CJK JP & Noto Serif CJK JP \\
和文ゴシック & Noto Sans CJK JP & Noto Sans CJK JP \\
\hline
\end{tabular}
\end{table}

\subsection{主要コマンドの実装}

\subsubsection{著者情報コマンド}

著者情報は以下の構造で管理する:

\begin{verbatim}
\RenewDocumentCommand{\author}{m m m o}{%
  \global\advance\@authornum by 1
  \IfValueTF{#4}{%
    \@addauthor{#1}{#2}{#3}{#4}%
  }{%
    \@addauthor{#1}{#2}{#3}{}%
  }%
}
\end{verbatim}

\subsubsection{図表参照コマンド}

IPSJ仕様の図表参照は以下で実現:

\begin{verbatim}
\newcommand{\figref}[1]{図\ref{#1}}
\newcommand{\tabref}[1]{表\ref{#1}}
\end{verbatim}

\section{評価実験}

\subsection{互換性評価}

既存のipsj.clsで作成された論文原稿30件について,ipsj-luatex.clsでの
処理を評価した結果を\figref{fig:compatibility}に示す.

\begin{figure}[tb]
\centering
\rule{7cm}{5cm}
\caption{互換性評価結果}
\ecaption{Compatibility evaluation results}
\label{fig:compatibility}
\end{figure}

93\%の原稿で追加修正なしでの処理が可能であった.残り7\%は主に以下の要因:
\begin{itemize}
\item 非標準パッケージの使用
\item pLaTeX固有の機能への依存
\end{itemize}

\subsection{性能評価}

コンパイル時間とメモリ使用量を従来手法と比較した結果を\tabref{tab:performance}に示す.

\begin{table}[tb]
\caption{性能比較結果}
\ecaption{Performance comparison results}
\label{tab:performance}
\centering
\begin{tabular}{|l|r|r|}
\hline
項目 & pLaTeX+ipsj.cls & LuaTeX+ipsj-luatex.cls \\
\hline\hline
コンパイル時間 & 2.3秒 & 3.1秒 \\
メモリ使用量 & 45MB & 67MB \\
PDF出力サイズ & 234KB & 245KB \\
\hline
\end{tabular}
\end{table}

LuaTeX環境では若干のオーバーヘッドがあるものの,実用上問題のない範囲である.

\section{おわりに}

本稿では,jlreqクラスをベースとしたLuaTeX用IPSJ論文クラスファイルipsj-luatex.clsを
開発した.評価実験により,従来のipsj.clsとの高い互換性を保ちながら,
LuaTeX環境の利点を活用した論文執筆環境を実現できることを確認した.

今後の課題として,以下が挙げられる:
\begin{itemize}
\item より多くの論文誌トランザクション固有機能の実装
\item パフォーマンスのさらなる最適化
\item 多言語対応の強化
\end{itemize}

\begin{acknowledgment}
本研究の実施にあたり,jlreq開発チームならびに情報処理学会論文誌編集委員会の
皆様に深く感謝いたします.
\end{acknowledgment}

\begin{thebibliography}{9}
\bibitem{ipsj-style}
情報処理学会:論文誌投稿スタイルファイル,
\url{https://www.ipsj.or.jp/journal/submit/style.html} (2025).

\bibitem{luatex-guide}
LuaTeX Development Team: LuaTeX Reference Manual,
\url{http://www.luatex.org/} (2025).

\bibitem{jlreq-w3c}
W3C: Requirements for Japanese Text Layout,
\url{https://www.w3.org/TR/jlreq/} (2020).

\bibitem{jlreq-class}
日本語組版処理の要件実装タスクフォース:jlreqクラス,
\url{https://github.com/abenori/jlreq} (2025).

\bibitem{latex3-project}
The LaTeX3 Project Team: The LaTeX3 Interfaces,
\url{https://www.latex-project.org/latex3/} (2024).

\bibitem{fontspec}
Robertson, W.: The fontspec package,
CTAN, \url{https://ctan.org/pkg/fontspec} (2024).

\bibitem{luatexja}
Japanese TeX Development Community: LuaTeX-ja,
\url{https://github.com/h-kitagawa/luatexja} (2025).

\bibitem{unicode-math}
Robertson, W. and Stephani, P.: The unicode-math package,
CTAN, \url{https://ctan.org/pkg/unicode-math} (2024).
\end{thebibliography}

\appendix
\section{実装の詳細}

\subsection{オプション一覧}

\tabref{tab:options}にサポートするオプションの一覧を示す.

\begin{table*}[t]
\caption{サポートオプション一覧}
\ecaption{List of supported options}
\label{tab:options}
\centering
\begin{tabular}{|l|l|l|}
\hline
カテゴリ & オプション & 説明 \\
\hline\hline
基本 & submit & 投稿用フォント設定 \\
 & english & 英文論文用 \\
 & technote & テクニカルノート \\
\hline
論文誌 & PRO & プログラミング \\
 & TOD & データベース \\
 & ACS & コンピューティングシステム \\
 & CDS & コンシューマ・デバイス\&システム \\
 & DCON & デジタルコンテンツ \\
 & TCE & 教育とコンピュータ \\
\hline
英文論文誌 & JIP & Journal of Information Processing \\
 & TBIO & Bioinformatics \\
 & CVA & Computer Vision and Applications \\
 & SLDM & System LSI Design Methodology \\
\hline
\end{tabular}
\end{table*}

\section{使用例}

基本的な使用例を以下に示す:

\begin{verbatim}
\documentclass[submit]{ipsj-luatex}
\usepackage{graphicx}

\title{論文表題}
\etitle{Paper Title}

\affiliate{ORG}{所属名\\Organization Name}
\author{著者名}{Author Name}{ORG}[email@example.com]

\jabstract{和文概要}
\jkeyword{キーワード1,キーワード2}
\eabstract{English abstract}
\ekeyword{keyword1, keyword2}

\begin{document}
\maketitle

\section{はじめに}
論文の内容...

\end{document}
\end{verbatim}

\end{document}